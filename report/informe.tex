\documentclass[11pt]{report}

% Paquetes y configuraciones adicionales
\usepackage{graphicx}
\usepackage[export]{adjustbox}
\usepackage{caption}
\usepackage{float}
\usepackage{titlesec}
\usepackage{geometry}
\usepackage[hidelinks]{hyperref}
\usepackage{titling}
\usepackage{titlesec}
\usepackage{parskip}
\usepackage{wasysym}
\usepackage{tikzsymbols}
\usepackage{fancyvrb}
\usepackage{xurl}
\usepackage{hyperref}
\usepackage[spanish]{babel}
\usepackage{listings}
\usepackage{subcaption}
\usepackage{xcolor}
\usepackage{amssymb}

\newcommand{\subtitle}[1]{
  \posttitle{
    \par\end{center}
    \begin{center}\large#1\end{center}
    \vskip0.5em}
}

% Configura los márgenes
\geometry{
  left=2cm,   % Ajusta este valor al margen izquierdo deseado
  right=2cm,  % Ajusta este valor al margen derecho deseado
  top=3cm,
  bottom=3cm,
}

% Configuración de los títulos de las secciones
\titlespacing{\section}{0pt}{\parskip}{\parskip}
\titlespacing{\subsection}{0pt}{\parskip}{\parskip}
\titlespacing{\subsubsection}{0pt}{\parskip}{\parskip}

% Redefinir el formato de los capítulos y añadir un punto después del número
\makeatletter
\renewcommand{\@makechapterhead}[1]{%
  \vspace*{0\p@} % Ajusta este valor para el espaciado deseado antes del título del capítulo
  {\parindent \z@ \raggedright \normalfont
    \ifnum \c@secnumdepth >\m@ne
        \huge\bfseries \thechapter.\ % Añade un punto después del número
    \fi
    \interlinepenalty\@M
    #1\par\nobreak
    \vspace{10pt} % Ajusta este valor para el espacio deseado después del título del capítulo
  }}
\makeatother

% Configura para que cada \chapter no comience en una pagina nueva
\makeatletter
\renewcommand\chapter{\@startsection{chapter}{0}{\z@}%
    {-3.5ex \@plus -1ex \@minus -.2ex}%
    {2.3ex \@plus.2ex}%
    {\normalfont\Large\bfseries}}
\makeatother

% Configurar los colores para el código
\definecolor{codegreen}{rgb}{0,0.6,0}
\definecolor{codegray}{rgb}{0.5,0.5,0.5}
\definecolor{codepurple}{rgb}{0.58,0,0.82}
\definecolor{backcolour}{rgb}{0.95,0.95,0.92}

% Configurar el estilo para el código
\lstdefinestyle{mystyle}{
  backgroundcolor=\color{backcolour},   
  commentstyle=\color{codegreen},
  keywordstyle=\color{magenta},
  numberstyle=\tiny\color{codegray},
  stringstyle=\color{codepurple},
  basicstyle=\ttfamily\footnotesize,
  breakatwhitespace=false,         
  breaklines=true,                 
  captionpos=b,                    
  keepspaces=true,                 
  numbers=left,                    
  numbersep=5pt,                  
  showspaces=false,                
  showstringspaces=false,
  showtabs=false,                  
  tabsize=2
}

\begin{document}

\title{CANARY ISLANDS DATABASE}
\author{Samuel Martín Morales  \texttt{alu0101359526@ull.edu.es} \and Jorge Domínguez González  \texttt{alu0101330600@ull.edu.es} \and Cheuk Kelly Ng Pante \texttt{alu0101364544@ull.edu.es} }
\date{\today}

\maketitle

\chapter{Objetivos del Proyecto}
El objetivo del proyecto es el diseño, creación e implementación de una base de datos para gestionar información relacionada con las Islas Canarias. 
La base de datos debe permitir realizar operaciones CRUD sobre la información almacenada en ella, así como consultas de prueba para demostrar su funcionamiento.


\chapter{Descripción del contexto de la base de datos}
La base de datos está destinada para almacenar información sobre las Islas Canarias, como por ejemplo, información sobre las islas, su distribución poblacional, compañías, sitios de interés y animales autóctonos.
La gestión de la información de las Islas Canarias es de gran importancia para el turismo, ya que permite a los turistas conocer mejor las islas y su historia, así como los lugares de interés que pueden visitar 
y los animales o plantas autóctonas que se pueden encontrar en nuestro archipiélago.

\section{Entidades:}
\begin{itemize}
    \item \textbf{Isla:}
    \subitem - Atributos: ID (clave primaria), Nombre.
    
    \item \textbf{Distribución Poblacional:}
    \subitem - Atributos: ID (clave primaria), Nombre, Provincia, Capital, Municipio, Poblacion Isla.
    
    \item \textbf{Compañías:}
    \subitem - Atributos: ID (clave primaria), Nombre, Tipo, Sede (relacionada con Islas), Año Fundacion.

    \item \textbf{Comestibles:}
    \subitem - Atributos: ID (clave primaria), Nombre, Tipo, Compañia.

    \item \textbf{Productos:}
    \subitem - Atributos: ID (clave primaria), Nombre, Islas.

    \item \textbf{Artesania:}
    \subitem - Atributos: ID (clave primaria), Nombre, Creador, Tipo.
    
    \item \textbf{Sitio interes:}
    \subitem - Atributos: ID (clave primaria), Nombre, Islas, Nombre Isla, Municipio, Latitud, Longitud,

    \item \textbf{Folklore:}
    \subitem - Atributos: ID (clave primaria), Nombre, Lanzamiento, Autor.

    \item \textbf{Seres Vivos:}
    \subitem - Atributos: ID (clave primaria), Nombre.
    
    \item \textbf{Animales autóctonos:}
    \subitem - Atributos: ID (clave primaria), Nombre, Nombre Cientifico, Islas, Invasoras, Dieta.

    \item \textbf{Plantas autóctonas:}
    \subitem - Atributos: ID (clave primaria), Nombre, Nombre Cientifico, Islas, Invasoras.

    \item \textbf{Nombres Canarios:}
    \subitem - Atributos: ID (clave primaria), Nombre, Isla, Genero.

    \item \textbf{Platos:}
    \subitem - Atributos: ID (clave primaria), Nombre, Tipo.

    \item \textbf{Ingredientes:}
    \subitem - Atributos: ID (clave primaria), Nombre.
\end{itemize}

Entre estas entidades nos encontramos con los siguientes tipos de entidades:
\begin{itemize}
    \item \textbf{Entidades Fuertes:} Isla, Compañia, Comestibles, Productos, Artesania, Seres Vivos, Nombres Canarios, Platos, Ingredientes.
    \item \textbf{Entidades Débiles:} Distribución Poblacional, Sitio interes, Folklore, Animales autóctonos, Plantas autóctonas.
\end{itemize}

\section{Relaciones:}
Las relaciones entre las entidades son las siguientes:
\begin{itemize}
    \item \textbf{Isla a Distribución Poblacional}
    \subitem - Relacionada por la columna Nombre con la entidad Islas.
    
    \item \textbf{Compañías a Islas:}
    \subitem - Relacionada por la columna Sede con la entidad Islas.

    \item \textbf{Sitio interes a Islas:}
    \subitem - Relacionada por la columna Isla con la entidad Islas.

    \item \textbf{Animales autóctonos a Islas:}
    \subitem - Relacionada por la columna Islas con la entidad Islas.
\end{itemize}

\section{Consideraciones Adicionales:}

\begin{itemize}
    \item La relación entre las entidades Compañías y Islas se establece a través de la columna Sede, indicando la isla donde tienen su sede las compañías.
    \item La relación entre Sitios Interes y Islas se establece por la columna Isla, indicando en qué isla se encuentra el sitio de interés.
    \item La relación entre Animales autóctonos e Islas se establece por la columna Islas, indicando las islas a las que están asociados los animales autóctonos.
\end{itemize}

\section{Restricciones:}

\begin{itemize}
    \item La relación entre las entidades Compañías y Islas se establece a través de la columna Sede, indicando la isla donde tienen su sede las compañías.
    \item La relación entre Sitio interes y Islas se establece por la columna Isla, indicando en qué isla se encuentra el sitio de interés.
    \item La relación entre Animales autóctonos e Islas se establece por la columna Islas, indicando las islas a las que están asociados los animales autóctonos.
\end{itemize}
\chapter{Diseño Conceptual}

\section{Modelo Entidad-Relación}
\begin{figure}[H]
    \centering
    \includegraphics[width=0.9\textwidth]{../diagrams/ER-PF-ADBD.jpg}
    \caption{Modelo Entidad-Relación}
    \label{fig:modelo_er}
\end{figure}

\section{Modelo Relacional}
\begin{figure}[H]
    \centering
    \includegraphics[width=0.9\textwidth]{../diagrams/RELACIONAL.jpg}
    \caption{Modelo Relacional}
    \label{fig:modelo_relacional}
\end{figure}

\section{Supuestos Semánticos}
Documentación que explique los supuestos semánticos y decisiones de diseño.

\chapter{Scripts SQL}
El script \emph{canary\_islands.sql} contiene la implementación de la base de datos en PostgreSQL. Para la ejecución
del script desde \emph{psql} se debe ejecutar el siguiente comando:
\begin{verbatim}
$ sudo -u postgres psql
postgres=# \i canary_islands.sql
\end{verbatim}

\section{Creación de la Base de Datos}
\begin{verbatim}
DROP DATABASE IF EXISTS islas_canarias;
CREATE DATABASE islas_canarias with TEMPLATE = template0 ENCODING = 'UTF8';

ALTER DATABASE islas_canarias OWNER TO postgres;

\connect islas_canarias

DROP SCHEMA IF EXISTS public CASCADE;
CREATE SCHEMA public;

ALTER SCHEMA public OWNER TO postgres;

SET default_tablespace = '';

SET default_table_access_method = heap;

\end{verbatim}

\section{Inicialización de las tablas}
Al ejecutar el script \emph{canary\_islands.sql} se crean las tablas de la base de datos, así como las relaciones entre ellas.
\begin{itemize}
    \item \textbf{Tabla Isla:} Es una tabla principal que representa cada isla del archipiélago canario.
    \lstset{style=mystyle}
    \lstinputlisting[language=sql]{src/table_isla.sql}

    \item \textbf{Tabla Seres Vivos:} 
    \lstset{style=mystyle}
    \lstinputlisting[language=sql]{src/table_seres_vivos.sql}

    \item \textbf{Tabla Animales Autoctonos:} 
    \lstset{style=mystyle}
    \lstinputlisting[language=sql]{src/table_animales_autoctonos.sql}

    \item \textbf{Tabla Plantas Autoctonas:}
    \lstset{style=mystyle}
    \lstinputlisting[language=sql]{src/table_plantas_autoctonas.sql}

    \item \textbf{Tabla Sitios Interes:}
    \lstset{style=mystyle}
    \lstinputlisting[language=sql]{src/table_sitios_interes.sql}

    \item \textbf{Tabla Distribución Poblacional:} 
    \lstset{style=mystyle}
    \lstinputlisting[language=sql]{src/table_distribucion_poblacional.sql}

    \item \textbf{Tabla Nombres Canarios:}
    \lstset{style=mystyle}
    \lstinputlisting[language=sql]{src/table_nombres_canarios.sql}

    \item \textbf{Tabla Platos:}
    \lstset{style=mystyle}
    \lstinputlisting[language=sql]{src/table_platos.sql}

    \item \textbf{Tabla Ingredientes:}
    \lstset{style=mystyle}
    \lstinputlisting[language=sql]{src/table_ingredientes.sql}

    \item \textbf{Tabla Comestibles:}
    \lstset{style=mystyle}
    \lstinputlisting[language=sql]{src/table_comestibles.sql}

    \item \textbf{Tabla Compania:}
    \lstset{style=mystyle}
    \lstinputlisting[language=sql]{src/table_compania.sql}

    \item \textbf{Tabla Artesania:}
    \lstset{style=mystyle}
    \lstinputlisting[language=sql]{src/table_artesania.sql}

    \item \textbf{Tabla Folclore:}
    \lstset{style=mystyle}
    \lstinputlisting[language=sql]{src/table_folclore.sql}

    \item \textbf{Tabla Isla Ecosistema:}
    \lstset{style=mystyle}
    \lstinputlisting[language=sql]{src/table_isla_ecosistema.sql}

    \item \textbf{Tabla Tejido Cultural:}
    \lstset{style=mystyle}
    \lstinputlisting[language=sql]{src/table_tejido_cultural.sql}

    \item \textbf{Tabla Plato Ingredientes:}
    \lstset{style=mystyle}
    \lstinputlisting[language=sql]{src/table_plato_ingredientes.sql}

    \item \textbf{Tabla Productos:}
    \lstset{style=mystyle}
    \lstinputlisting[language=sql]{src/table_productos.sql}

    \item \textbf{Tabla Produccion Compañia:}
    \lstset{style=mystyle}
    \lstinputlisting[language=sql]{src/table_produccion_compania.sql}

    \item \textbf{Tabla Distribución Gastronomica:}
    \lstset{style=mystyle}
    \lstinputlisting[language=sql]{src/table_distribucion_gastronomica.sql}
\end{itemize}


\section{Inclusión de Datos en las Tablas}
\begin{verbatim}
-- -- Inclusión de datos en la tabla de isla_ecosistema
ALTER TABLE isla_ecosistema
ALTER COLUMN plantas_autoctonas_id DROP NOT NULL,
ALTER COLUMN animales_autoctonos_id DROP NOT NULL;

TRUNCATE TABLE isla_ecosistema;

INSERT INTO isla_ecosistema(isla_id, seres_vivos_id, animales_autoctonos_id, plantas_autoctonas_id)
SELECT isla_id, ser_vivo_id, id_animales_autoctonos, NULL
FROM animales_autoctonos;

INSERT INTO isla_ecosistema(isla_id, seres_vivos_id, animales_autoctonos_id, plantas_autoctonas_id)
SELECT isla_id, ser_vivo_id, NULL, id_plantas_autoctonas
FROM plantas_autoctonas;
\end{verbatim}

\section{Implementación de Triggers}
\begin{verbatim}
-- Si se añade una nueva tupla dentro de la tabla de animales_autoctonos, se añadirá una nueva tupla en la tabla de isla_ecosistema
CREATE OR REPLACE FUNCTION insertar_animal_autoctono() RETURNS TRIGGER AS $$
BEGIN
    INSERT INTO isla_ecosistema(isla_id, seres_vivos_id, animales_autoctonos_id, plantas_autoctonas_id)
    VALUES (NEW.isla_id, NEW.ser_vivo_id, NEW.id_animales_autoctonos, NULL);
    RETURN NEW;
END;
$$ LANGUAGE plpgsql;

CREATE TRIGGER insertar_animal_autoctono
AFTER INSERT ON animales_autoctonos
FOR EACH ROW
EXECUTE PROCEDURE insertar_animal_autoctono();
\end{verbatim}

\chapter{Consultas de Ejemplo}

\section{Consultas SQL}
Ejemplos de consultas que demuestren el funcionamiento de la base de datos.

\chapter{Implementación de API con Flask}

\section{API REST}
Desarrollo de una API mediante Flask para realizar operaciones CRUD.

\chapter{Entrega}

\section{Repositorio en GitHub}
Enlace al Repositorio: \url{https://github.com/feichay10/Proyecto-Final-ADBD}

\section{Imágenes Adjuntas}
Modelo Entidad-Relación, Grafo Relacional y capturas de consultas y operaciones en las tablas.

\chapter{Bibliografía}
\begin{thebibliography}{99}
    \bibitem{1} \url{https://www.canaryislands.org/}

\end{thebibliography}

\end{document}
